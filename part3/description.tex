% Created 2017-04-16 Sun 15:47
\documentclass[11pt]{article}
\usepackage[utf8]{inputenc}
\usepackage[T1]{fontenc}
\usepackage{fixltx2e}
\usepackage{graphicx}
\usepackage{longtable}
\usepackage{float}
\usepackage{wrapfig}
\usepackage{rotating}
\usepackage[normalem]{ulem}
\usepackage{amsmath}
\usepackage{textcomp}
\usepackage{marvosym}
\usepackage{wasysym}
\usepackage{amssymb}
\usepackage{hyperref}
\tolerance=1000
\usepackage{listings}
\author{Weston Ortiz, Hans Weeks}
\date{April 16, 2017}
\title{Description of Queries}
\hypersetup{
  pdfkeywords={},
  pdfsubject={},
  pdfcreator={Emacs 25.1.1 (Org mode 8.2.10)}}
\begin{document}

\maketitle
\tableofcontents


\section{Data Retrieval Queries}
\label{sec-1}

\begin{enumerate}
\item How many amendments were introduced before 2008 in each branch?
\begin{verbatim}
select type, count(type) from Amendment where 
introduced_at < '2008' group by type;
\end{verbatim}

\item Which legislators were born before the outset of WWI?
\begin{verbatim}
select distinct bioguide_id, `First Name` , `Last Name`, party 
from
Legislator natural join Term where birthday < '1914-07-28';
\end{verbatim}

\item Find all Legislators that shared a birthday
\begin{verbatim}
select L1.birthday, L1.bioguide_id, L1.`First Name`, L1.`Last Name`,
  L2.bioguide_id, L2.`First Name`, L2.`Last Name` 
from Legislator L1, Legislator L2 
where L1.birthday = L2.birthday and L1.bioguide_id != L2.bioguide_id;
\end{verbatim}

\item Find all votes that have passed
\begin{verbatim}
select * from Vote where result = 'PASSED';
\end{verbatim}

\item How many legislators have there been for each party 
(using only a legislators most recent term to decide their party)
\begin{verbatim}
select party, count(party) as count 
from Term join (select bioguide_id, max(start) as ms 
		from Term GROUP BY bioguide_id) as latest 
ON Term.start = latest.ms and Term.bioguide_id = latest.bioguide_id 
GROUP BY party ORDER BY count DESC;
\end{verbatim}

\item For a Bill find all congress members and how they voted on the latest vote for that bill
\begin{verbatim}
select `bioguide_id`, `First Name`, `Last Name`, how_voted 
from (select * from Legislator natural join Legislator_Vote) as LV 
     join Vote ON Vote.id = LV.Vote_id 
where Vote_id = (select id from Vote where Bill_id =  'hr899-113' 
      and number = (select max(number) from Vote where Bill_id = 'hr899-113'));
\end{verbatim}

\item Find all bills sponsored by a legislator from NM and the name of the
legislator that sponsored it.
\begin{verbatim}
select `Last Name`, Bill_id  
from ((select bioguide_id from Term where state ='NM') as t1 
     natural join Legislator) natural join Sponsor;
\end{verbatim}

\item Union the shared fields from amendment and bill, represents all pieces
of legislation in the database:
\begin{verbatim}
select * 
from (select id, type, status, introduced_at as introduction_date, 
      congress, number 
      from Amendment) as t1 
     union 
     (select id, type, status, introduction_date, 
	     congress, number from Bill) limit 10;
\end{verbatim}

\item For a given vote count the number of yes votes for each party
\begin{verbatim}
select party, count(party) as count 
from (select distinct bioguide_id, party 
      from Legislator_Vote natural join Term 
      where Vote_id = 'h136-115.2017' and how_voted='Yea') as yes 
GROUP BY party;
\end{verbatim}

\item List the members of congress for congress 115 and order by state
\begin{verbatim}
select distinct `First Name`, `Last Name`, state 
from Legislator natural join Term 
where start >= (select begin from Congress where id = 115) 
      and end <= (select end from Congress where id = 115) ORDER BY state
\end{verbatim}

\item Find all subjects for Congress 114 (would be useful for navigation in the web interface)
\begin{verbatim}
select distinct subject from Subject join Bill ON Subject.Bill_id = Bill.id 
where congress = 114;
\end{verbatim}

\item Count the number of terms a member of congress has served 
(only if we have a record of them voting in our data)
\begin{verbatim}
select bioguide_id, `First Name`, `Last Name`, Terms 
from (select bioguide_id, count(bioguide_id) as Terms 
      from Term GROUP BY bioguide_id) as tm natural join Legislator 
where bioguide_id in (select bioguide_id from Legislator_Vote) 
ORDER BY Terms DESC;
\end{verbatim}

\item Find all Legislators that share a last name
\begin{verbatim}
select L1.bioguide_id, L1.`First Name`, L1.`Last Name`,
       L2.bioguide_id, L2.`First Name`, L2.`Last Name` 
from Legislator L1, Legislator L2 
where L1.`Last Name` = L2.`Last Name` 
      and L1.bioguide_id != L2.bioguide_id;
\end{verbatim}

\item Find all the ways that have been voted
\begin{verbatim}
select distinct how_voted from Legislator_Vote;
\end{verbatim}
\end{enumerate}

\section{Modification Queries}
\label{sec-2}


\begin{enumerate}
\item If a Legislator does not have a full name update it with 
their wikipedia id (which is a full name) only if 
their wikipedia id is not null as well
\begin{verbatim}
update Legislator 
set official_full_name=wikipedia_id 
where official_full_name IS NULL and wikipedia_id IS NOT NULL;
\end{verbatim}

\item Change a legislators end date for a specific Term
(like if they have been removed from office, 
 in our case Tom Marino is likely to be appointed to another position soon)
\begin{verbatim}
update Term
set end = '2017-05-01'
where bioguide_id = (select bioguide_id from Legislator 
		     where `First Name` = 'Tom' and `Last Name` = 'Marino') 
		     and end > '2017-05-01';
\end{verbatim}

\item Delete all Legislators , Sponsor, and Terms for those legislator 
if they never participated in a vote in our data
\begin{verbatim}
delete from Term 
where bioguide_id 
      not in (select bioguide_id from Legislator_Vote);

delete from Sponsor 
where Legislator_id 
      not in (select bioguide_id from Legislator_Vote);

delete from Legislator 
where bioguide_id 
      not in (select bioguide_id from Legislator_Vote);
\end{verbatim}

\item Delete all votes that we do not have a record of Legislators voting on it
\begin{verbatim}
delete from Vote 
where id 
      not in (select Vote_id as id from Legislator_Vote);
\end{verbatim}
\end{enumerate}


\begin{enumerate}
\item Add a new Legislator (a special election has occured and Ron Estes needs to be added).
\begin{verbatim}
insert into Legislator (bioguide_id, `Last Name`, `First Name`, 
                        birthday, gender, `wikipedia_id`, 
                        govtrack_id, official_full_name)     
Value     
('E000298', 'Estes', 'Ron', '1956-07-19', 'M', NULL, 412735, NULL);
\end{verbatim}

\item Add a new term for that legislator (each legislator should have a corresponding Term)
\begin{verbatim}
insert into Term 
VALUE ('E000298', '2017-04-27', '2019-01-03', 
       'rep', 'KS', NULL, 4, 'Republican', 'h');
\end{verbatim}
\end{enumerate}

\section{Useful Indexes}
\label{sec-3}

\begin{enumerate}
\item The query where we find all legislators that share a birthday was quite slow.
We increased the speed of this dramatically by creating an index on their birthday.
\begin{verbatim}
create index bday on Legislator (birthday);
\end{verbatim}

\item When finding all pairs of legislators with the same last name speed was
imporved by adding a index on `Last Name` column
\begin{verbatim}
create index lastName on Legislator (`Last Name`);
\end{verbatim}

\item When counting unique ways in which legislators have voted the query speed
was increased by added an index on the corresponding column:
\begin{verbatim}
create index voteindex on Legislator_Vote(how_voted);
\end{verbatim}
\end{enumerate}

\section{Modifying data from the last project}
\label{sec-4}

\begin{itemize}
\item For Legislator: First Name, Last name were the same due to an script typo.

\item For Term: Party is Unknown for all members.

\item For bill and amendments there are ‘None’ values instead of NULL.
\end{itemize}

All of this was fixed in our github repo and in our database: \url{https://github.com/hhweeks/congress_db/}

Term and Legislator were updated by correcting the script, Bill and Amendments were corrected by
performing a sed replace on 'None' to NULL in the correct sql import files.
% Emacs 25.1.1 (Org mode 8.2.10)
\end{document}
